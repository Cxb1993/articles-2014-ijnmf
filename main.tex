\typeout{ ====================================================================}
\typeout{ this is file main.tex, created at 27-Jan-2014                       }
\typeout{ maintained by Gustavo Rabello dos Anjos                             }
\typeout{ e-mail: gustavo.rabello@gmail.com                                   }
\typeout{ ====================================================================}


\documentclass[12pt,fleqn]{article}
\usepackage[T1]{fontenc}              % acentuacao e hifenizacao portuguesa
\usepackage[hmargin=30mm,vmargin=25mm]{geometry}
%\usepackage[pdftex]{graphicx}        % pacote para inclusao de figuras,
\usepackage{subfloat}                 % pacote para figuras duplas (subfigure)
\usepackage{amssymb,amsmath,amsfonts} % pacote matematicos
\usepackage{tocbibind}		      % para inclusao de bib e ind na TOC
\usepackage{pictexwd,color}	          % ambiente para PICTEX
\usepackage{alltt,verbatim,moreverb}  % para entrada verbatim comandos LaTeX
\usepackage{amscd}		              % para setas longas
\usepackage{GCPOTools}                % GCPO's UDFs


\begin{document}

	\title{Two-phase flows in sinusoidally constricted channel using
	       moving mesh/boundary technique}

	\author{\rm
		\scriptsize
		\begin{tabular}{l}
			\textbf{Gustavo Anjos} -- gustavo.rabello@gmail.com\\
			Mechanical Engineering Department/GESAR Group, State University of Rio de Janeiro\\
			R. S\~ao Francisco Xavier 524, 20550-013 Rio de Janeiro, RJ, Brazil\\
			\textbf{Gustavo Oliveira} -- tavolesliv@gmail.com\\
			Mechanical Engineering Department/GESAR Group, State University of Rio de Janeiro\\
			R. S\~ao Francisco Xavier 524, 20550-013 Rio de Janeiro, RJ, Brazil\\
			\textbf{Jos\'e Pontes} -- jopontes@metalmat.ufrj.br\\
			Escola Polit\'ecnica/COPPE -- Universidade Federal do Rio de Janeiro\\
			P.O. Box 68505, Rio de Janeiro, RJ, 21941-972 Brasil\\
			\textbf{John Thome} -- john.thome@epfl.ch\\
			EPFL STI IGM LTCM, ME B1 345, Station 9, CH-1015, Lausanne, Switzerland\\
			\textbf{Norberto Mangiavacchi} -- norberto.mangiavacchi@gmail.com\\
			Mechanical Engineering Department/GESAR Group, State University of Rio de Janeiro\\
			R. S\~ao Francisco Xavier 524, 20550-013 Rio de Janeiro, RJ, Brazil
		\end{tabular}
	}	

\maketitle
\thispagestyle{empty}
\begin{abstract}
Bubbles and drops dynamics through capillaries of variable cross-section
still remains of considerable importantce in two-phase flows through
porous media. Experimental studies are found in the literature where the
motion of immiscible bubbles in different fluids is investigated.
Recovery of oil by chemical flooding, biological processes and crude oil
transportation in through pipelines are exemples of industry-related
applications. We seek to study numerically the effects of the surface
tension, bubble dynamics and channel geometry for two-phase flows in
sinusoidally constricted channels using a moving boundary domain scheme,
which dramatically shortens the domain length. Such a scheme moves the
computational boundary nodes periodically according to the flow field or
bubble centroid's velocity. The set of equations is written in a
generalized form namely the Arbitrary Lagrangian-Eulerian (ALE)
description, which combines the best aspects of both Lagrangian and
Eulerian framework. The two-phase inteface position moves according to
the flow field and it is explicity described by a set of interconnected
nodes, segments and elements which ensures a sharp representation of the
front, not requiring the use of any additional equation of motion
\cite{anjos2012},\cite{anjos2014}. Unlike the traditional numerical
approach where the domain's boundary is fixed in space and time, the
boundary nodes are in constant motion, thus simulating the relative
velocity between the bubble and the wall. Such a technique allows one to
decrease considerable the domain length and, therefore, decrease the
computational processing time. The new methodology proposed to simulate
two-phase flows in sinusoidally constricted channels is compared to
experimental results found in the literature
\cite{olbricht1983},\cite{hemmat1996} showing good accuracy to describe
interfacial forces and bubble dynamics in different complex geometries
with moving boundaries. 
\end{abstract}

%--------------------------------------------------
% \keywords{\em{Sinusoidally constricted channel, Moving boundary,
% Two-phase flows, Arbitrary Lagrangian-Eulerian, Finite Element Method,
% complex geometry}}
%-------------------------------------------------- 


\section{Introduction}

Capillaries with variable cross-sections are used in enginnering
applications to incite unsteady behaviours in the kinematics of flows, as
opposed to the uniformity borne out in constant cross-sections
domains. Some examples of two-phase flow modeling in diverse geometries,
for instance, have been applied to support studies upon enhanced oil
recovery (EOR) techniques by chemical flooding \cite{Olbricht1996},
pore-scale prototyping and mobility of immiscible fluids in porous media of 
subsurface \cite{Hemmat1996}, and alteration of blood flow caused by
vascular occlusions \cite{Forsester1970}. In such problems and like ones,
converging/diverging tubes, sinusoidally constricted capillaries, as
well as domains formed by a periodic pace of abrupt changes of diameters
are extensively taken into account in their mathematical modeling.

A relevant experimental insight about the influence of a tube with
wavy-wall geometry upon the dynamics of a droplet motion was given by
\cite{OlbrichtAndLeal1983}, whereby they concluded that it depends
strongly on the capillary number. Moreover, they found that the pressure
drop of the droplet is always associated to the constrictions of the
tube, which, for a real porous media, would be tied to the retention of
bubbles at determined sites and their consequent low mobility.
%Comparatively, \cite{Graham2000} presented  




\section{Governing Equations}
\section{Results}
\section{Conclusions}

\bibliographystyle{plain}
\bibliography{bibliography}

\end{document}





\typeout{ ****************** End of file main.tex ****************** }

