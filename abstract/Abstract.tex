\documentclass{Eccomas}
\usepackage{graphicx}
\usepackage{amsmath}
\usepackage{amsfonts}
\usepackage{amssymb}

\title{Two-phase flows in sinusoidally constricted channel using moving
mesh/boundary technique}

\author{Gustavo Anjos$^{1}$, Gustavo C.P. Oliveira$^{1}$, Jose Pontes$^{2}$
and John R. Thome$^{3}$, Norberto Mangiavacchi$^{1}$}

\heading{Gustavo Anjos, Gustavo Oliveira, Jose Pontes, John Thome and
Norberto Mangiavacchi}

\address{$^{1}$ Departamento de Engenharia Mec\^anica, Universidade do
Estado do Rio de Janeiro, Rua S\~ao Francisco Xavier, 524, 20550-900,
Rio de Janeiro, RJ, Brazil,  gustavo.rabello@gmail.com,
http://gustavo.rabello.org\\
$^{2}$ Metallurgy and Materials Engineering Dept - PEMM/COPPE/UFRJ, P.O. Box
68505, 21941-972, Rio de Janeiro R. J., Brazil,  jopontes@metalmat.ufr.br\\
$^{3}$ EPFL STI IGM LTCM, ME B1 345, Station 9, CH-1015, Lausanne,
Switzerland,  john.thome@epfl.ch}

\keywords{Sinusoidally constricted channel, Moving boundary, Two-phase
flows, Arbitrary Lagrangian-Eulerian, Finite Element Method, complex
geometry }

\begin{document}

Bubbles and drops dynamics through capillaries of variable cross-section
still remains of considerable importantce in two-phase flows through
porous media. Experimental studies are found in the literature where the
motion of immiscible bubbles in different fluids is investigated.
Recovery of oil by chemical flooding, biological processes and crude oil
transportation in through pipelines are exemples of industry-related
applications. 

We seek to study numerically the effects of the surface tension, bubble
dynamics and channel geometry for two-phase flows in sinusoidally
constricted channels using a moving boundary domain scheme, which
dramatically shortens the domain length. Such a scheme moves the
computational boundary nodes periodically according to the flow field or
bubble centroid's velocity. The set of equations is written in a
generalized form namely the Arbitrary Lagrangian-Eulerian (ALE)
description, which combines the best aspects of both Lagrangian and
Eulerian framework. The two-phase inteface position moves according to
the flow field and it is explicity described by a set of interconnected
nodes, segments and elements which ensures a sharp representation of the
front, not requiring the use of any additional equation of motion
\cite{anjos2012},\cite{anjos2014}. Unlike the traditional numerical
approach where the domain's boundary is fixed in space and time, the
boundary nodes are in constant motion, thus simulating the relative
velocity between the bubble and the wall. Such a technique allows one to
decrease considerable the domain length and, therefore, decrease the
computational processing time. 

The new methodology proposed to simulate two-phase flows in sinusoidally
constricted channels is compared to experimental results found in the
literature \cite{olbricht1983},\cite{hemmat1996} showing good accuracy
to describe interfacial forces and bubble dynamics in different complex
geometries with moving boundaries. 

%
%   Finally the references done as a simple list,
%   rather than by a redefined bibliography environment...
%
%\vspace{0.5cm}

\begin{thebibliography}{99}
\setlength{\parskip}{0pt}

\bibitem{anjos2012}
G.R. Anjos.
\newblock {\em A 3D ALE Finite Element Method for Two-Phase Flows with Phase
  Change}.
\newblock PhD thesis, \'Ecole Polytechnique F\'ed\'erale de Lausanne, July
  2012.

\bibitem{anjos2014}
G.R. Anjos, N.~Borhani, N.~Mangiavacchi, and J.R. Thome.
\newblock 3d ale finite element method for two-phase flows.
\newblock {\em Journal of Computational Physics}, 2014.
\newblock Accepted for publication.

\bibitem{hemmat1996}
M.~Hemmat and A.~Borhan.
\newblock Buoyancy-driven motion of drops and bubbles in a periodically
  constricted capillary.
\newblock {\em Chemical Engineering Communications}, 148-150(1):363--384, 1996.

\bibitem{olbricht1983}
W.L. Olbricht and L.G. Leal.
\newblock The creeping motion of immiscible drops through a
  converging/diverging tube.
\newblock {\em Journal of Fluid Mechanics}, 134(1):329--355, 1983.

%\bibliographystyle{plain}
%bibliography{/Users/gustavo/projects/misc/latex/referencias}

\end{thebibliography}

\end{document}


