\typeout{ ====================================================================}
\typeout{ this is file main.tex, created at 27-Jan-2014                       }
\typeout{ maintained by Gustavo Rabello dos Anjos                             }
\typeout{ e-mail: gustavo.rabello@gmail.com                                   }
\typeout{ ====================================================================}


\documentclass[12pt,fleqn]{article}
\usepackage[T1]{fontenc}              % acentuacao e hifenizacao portuguesa
\usepackage[hmargin=30mm,vmargin=25mm]{geometry}
%\usepackage[pdftex]{graphicx}        % pacote para inclusao de figuras,
\usepackage{subfloat}                 % pacote para figuras duplas (subfigure)
\usepackage{amssymb,amsmath,amsfonts} % pacote matematicos
\usepackage{tocbibind}		      % para inclusao de bib e ind na TOC
\usepackage{pictexwd,color}	          % ambiente para PICTEX
\usepackage{alltt,verbatim,moreverb}  % para entrada verbatim comandos LaTeX
\usepackage{amscd}		              % para setas longas
%\usepackage{GCPOTools}                % GCPO's UDFs


\begin{document}

	\title{Two-phase flows in sinusoidally constricted channel using
	       moving mesh/boundary technique}

	\author{\rm
		\scriptsize
		\begin{tabular}{l}
			\textbf{Gustavo Anjos} -- gustavo.anjos@uerj.br\\
			Mechanical Engineering Department/GESAR Group, State University of Rio de Janeiro\\
			R. Fonseca Teles 121, 20550-013, Rio de Janeiro, RJ, Brazil\\
			\textbf{Gustavo Oliveira} -- gustavo.oliveira@uerj.br\\
			Mechanical Engineering Department/GESAR Group, State University of Rio de Janeiro\\
			R. Fonseca Teles 121, 20550-013 Rio de Janeiro, RJ, Brazil\\
			\textbf{Jos\'e Pontes} -- jopontes@metalmat.ufrj.br\\
			Escola Polit\'ecnica/COPPE -- Universidade Federal do Rio de Janeiro\\
			P.O. Box 68505, Rio de Janeiro, RJ, 21941-972 Brasil\\
			\textbf{John Thome} -- john.thome@epfl.ch\\
			EPFL STI IGM LTCM, ME B1 345, Station 9, CH-1015, Lausanne, Switzerland\\
			\textbf{Norberto Mangiavacchi} -- norberto@uerj.br\\
			Mechanical Engineering Department/GESAR Group, State University of Rio de Janeiro\\
			R. Fonseca Teles 121, 20550-013, Rio de Janeiro, RJ, Brazil
		\end{tabular}
	}	

\maketitle
\thispagestyle{empty}
\begin{abstract}
Bubbles and drops dynamics through capillaries of variable cross-section
still remains of considerable importantce in two-phase flows through
porous media. Experimental studies are found in the literature where the
motion of immiscible bubbles in different fluids is investigated.
Recovery of oil by chemical flooding, biological processes and crude oil
transportation in through pipelines are exemples of industry-related
applications. We seek to study numerically the effects of the surface
tension, bubble dynamics and channel geometry for two-phase flows in
sinusoidally constricted channels using a moving boundary domain scheme,
which dramatically shortens the domain length. Such a scheme moves the
computational boundary nodes periodically according to the flow field or
bubble centroid's velocity. The set of equations is written in a
generalized form namely the Arbitrary Lagrangian-Eulerian (ALE)
description, which combines the best aspects of both Lagrangian and
Eulerian framework. The two-phase inteface position moves according to
the flow field and it is explicity described by a set of interconnected
nodes, segments and elements which ensures a sharp representation of the
front, not requiring the use of any additional equation of motion
\cite{anjos2012},\cite{anjos2014}. Unlike the traditional numerical
approach where the domain's boundary is fixed in space and time, the
boundary nodes are in constant motion, thus simulating the relative
velocity between the bubble and the wall. Such a technique allows one to
decrease considerable the domain length and, therefore, decrease the
computational processing time. The new methodology proposed to simulate
two-phase flows in sinusoidally constricted channels is compared to
experimental results found in the literature
\cite{olbricht1983},\cite{hemmat1996} showing good accuracy to describe
interfacial forces and bubble dynamics in different complex geometries
with moving boundaries. 
\end{abstract}

%--------------------------------------------------
% \keywords{\em{Sinusoidally constricted channel, Moving boundary,
% Two-phase flows, Arbitrary Lagrangian-Eulerian, Finite Element Method,
% complex geometry}}
%-------------------------------------------------- 


\section{Introduction}

Capillaries with variable cross-sections are used in enginnering
applications to incite unsteady behaviours in the kinematics of flows,
as opposed to the uniformity borne out in constant cross-sections
domains. Two-phase flow modelling in diverse geometries, were focused on 
studies upon enhanced oil recovery (EOR) techniques by chemical 
flooding \cite{olbricht1996}, \cite{cobos2009}, pore-scale prototyping, 
evaluation of mobility control of immiscible fluids in porous media of 
subsurface \cite{hemmat1996}, and alteration of blood flow caused by 
vascular occlusions \cite{forrester1970}, for instance. In such problems
and like ones, converging/diverging tubes, sinusoidally constricted capillaries, as well as 
domains formed by a periodic pace of throats are extensively taken into account 
for their mathematical modelling.

A significant experimental insight about the influence of a tube with
wavy-wall geometry upon the dynamics of a droplet motion was given by
\cite{olbricht1983}, whereby they concluded that it depends
strongly on the capillary number. Moreover, they found that the pressure
drop of the droplet is always associated to the constrictions of the
tube, which, for a real porous media, would be tied to the retention of
bubbles at determined sites and their consequent low mobility.
Comparatively, the pore plugging phenomenon, by which trapped droplets
hinder the overall permeability of the medium, was faced by
\cite{graham2000} through acoustic stimulation. Their results showed
efficiency as to remobilization processes and good capabilities for
application in several industrial operations, such as filtration issues,
flow in packed beds, and the manufacture of fibrous composites. Drop
breakup in constricted capillaries, or \emph{snap-off} effects, is 
an appreciated issue \cite{tsai1994}. Apart
from these, recent research was also carried out in diverging/converging geometries to
examine heat transfer characteristics of suspensions and polymers
\cite{narayanam2014}, which awakens a wide range of interest upon these
flows.

With attention to computational works propping up the investigation of
the motion of disperse elements through narrow passages and bottlenecks,
techniques relied on finite volume/front-tracking were introduced in the
modelling of buoyancy-driven viscous drop through constricted
capillaries, where the interfaces were represented by using connected
Lagrangian marker points which move with the local flow velocity
\cite{muradoglu2006}, \cite{olgac2006}. More
recently, implementations gathering finite element/level-set methods
also were performed in order to elucidate the dynamics of drops inside
constricted microcapillaries headed to industrial purposes as regards
oil-water emulsions, mobility control, and pressure drop, for instance
\cite{roca2013}. 

In this work, we seek to study numerically the effects of surface
tension, bubble dynamics, and geometry on two-phase flows in
sinusoidally constricted channels by using a moving boundary domain
scheme, which shortens the domain length dramatically. Such scheme moves
the computational boundary nodes periodically according to the flow
field or the bubble centroid's velocity. The set of equations is written
in a generalized form, namely the Arbitrary Lagrangian-Eulerian (ALE)
description, so combining the best aspects of both Lagrangian and
Eulerian descriptions. To model the interface between the phases, we
allow that the points over it move according to the flow field. In turn,
the interface is explicitly described by a set of interconnected nodes,
segments, and elements so ensuring a sharp representation of the front
and supplanting the use of any additional equation to govern the movement
of interface points \cite{anjos2012}, \cite{anjos2014}. Unlike the
traditional numerical approach where the domain's boundary is fixed in
space and time, the boundary nodes are kept in constant motion, thus
simulating the relative velocity between the bubble and the wall.
Moreover, the results are obtained over a reduced computational domain,
which, therefore, implies a smaller computational processing time.

The paper is organized as follows: Section \ref{sec:governing} exposes
briefly the set of equations governing the physical problem; Section
\ref{sec:ale} gives an overview of the ALE formulation implemented; 
Section \ref{sec:results} introduces the results obtained, and, in last, 
conclusions are drawn in Section \ref{sec:conclusion}. 


\section{{\label{sec:governing}Governing Equations}
The non-dimensional incompressible conservation equations for two-phase
flows is presented below through the generalized Arbitrary
Lagrangian-Eulerian (ALE) framework including the surface tension term
$\mathbf{f}$ and the gravity term $\mathbf{g}$: 

\begin{equation}
	\rho(\phi) [ \frac{\partial \uvet}{\partial t} 
	+ \cvet \cdot \nabla \uvet ]
	=
	- \nabla p 
	+ \frac{1}{\N^{1/2}} \nabla \cdot
	[\mu(\phi) ( \nabla \uvet + \nabla \uvet^T)]+
	\rho(\phi) \mathbf{g}+
	\frac{1}{\Eo} \mathbf{f}
	\label{eq:nd-momentum}
\end{equation}

\begin{equation}
	\nabla \cdot \uvet
	= 
	0.
\label{eq:nd-continuity} 
\end{equation}

On the left hand side of Eq.~\ref{eq:nd-momentum}, the convective velocity
$\cvet$ represents the relative velocity between the flow field and the
mesh, given by the following expression: $\cvet = \uvet - \hat{\uvet}$
where $\uvet$ stands for the computed flow field velocity and
$\hat{\uvet}$ for the mesh velocity. Pressure is represented by $p$ and
time by $t$. The fluid density $\rho(\phi)$ and viscosity $\mu(\phi)$
are defined as constant in each phase. The non-dimensional groups
\textit{Archimedes} and \textit{E\"otv\"os} are defined as follow:

\begin{equation}
 \N = \frac{\rho_0^2 D^3 g}{\mu_0^2},
 \Eo = \frac{\rho_0 g D^2}{\sigma_0},
 \Mo = \frac{(\rho_l-\rho_g)\mu_0^4 g}{\rho_{0}^2 \sigma_0^3} 
 \label{eq:nd-groups}
\end{equation}

section{Results}
\section{Conclusions}

\bibliographystyle{plain}
\bibliography{bibliography}

\end{document}





\typeout{ ****************** End of file main.tex ****************** }

